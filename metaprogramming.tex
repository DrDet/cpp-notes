\section{Метапрограммирование в C++11 и выше}
\subsection{SFINAE для выражений}
Одним из самых значительных улучшений SFINAE в C++11 стало появление SFINAE для выражений (Expression SFINAE). \\\\
Его суть заключается в том, что ошибки подстановки выражений в сигнатуре функции трактуется как ошибки SFINAE, а не как hard-ошибки.
В C++03 гарантировалось лишь то, что ошибки подстановки только в \textit{constant expression'ы} трактуются как ошибки SFINAE. Ошибки подстановки в не-constant expression'ы могли в зависимости от реализации трактоваться и как SFINAE-ошибки, и как hard-ошибки. \\\\
Часто SFINAE для выражений используется вместе с оператором запятая. В невычисляемом левом аргументе оператора запятая можно перечислить выражения, которые должны участвовать в SFINAE проверках. \\\\
Ниже приведен пример функции \texttt{void print(T)}, первая перегрузка которой умеет выводить все типы, имеющие функции cbegin() и cend(), вторая --- любые типы, умеющие выводиться в std::cout.
\begin{minted}[linenos, frame=lines, framesep=2mm, tabsize = 4, breaklines]{c++}
template <typename T>
decltype(std::declval<T>().cbegin(), std::declval<T>().cend(), void())
print(T container)
{
	std::cout << "Values:{ ";
	for(auto value : container)
	std::cout << value << " ";
	std::cout << "}\n";
}

template <typename T>
decltype(std::cout << std::declval<T>(), void())
print(T value)
{
	std::cout << "Value: " << value << "\n";
}	
\end{minted}
\subsection{Default template arguments in function templates \& SFINAE}
В С++11 появилась возможность устанавливать значения параметров шаблона функций по умолчанию. \\\\
Вывод типов в шаблонах функций с аргументами по умолчанию производится следующим образом:
\begin{enumerate}
	\item Компилятор пытается вывести тип всех параметров шаблона.
	\item Если какой-то параметр вывести не удалось, то используется его значение по умолчанию. Если значение по умолчанию отсутствует --- происходит ошибка вывода.
\end{enumerate}
Небольшой пример для лучшего понимания:
\begin{minted} [linenos, frame=lines, framesep=2mm, tabsize = 4, breaklines]{c++}
template <class T, class U = double>
void f(T t = 0, U u = 0);

char c;				//any symbol

void g()
{
	f(1, c);           // f<int,char>(1,c)
	f(1);              // f<int,double>(1,0)
	f();               // error: T cannot be deduced
	f<int>();          // f<int,double>(0,0)
	f<int,char>();     // f<int,char>(0,0)
}
\end{minted}
Поскольку значение по умолчанию для параметра шаблона может зависеть от других параметров шаблона, для его вычисления делается подстановка, которая может приводить к ошибкам подстановки. В C++11 требуется, чтобы ошибки подстановки в значение по умолчанию параметра шаблона трактовались как ошибки SFINAE. Это нововведение можно использовать для наложения ограничений на типы. \\\\
Рассмотрим в качестве примера реализацию функции модуля для знаковых чисел:
\begin{minted} [linenos, frame=lines, framesep=2mm, tabsize = 4, breaklines]{c++}
template <typename T, 
		  typename = enable_if<is_signed<T>::value>>::type>
T abs(T val);
\end{minted}
Второй шаблонный параметр функции \texttt{\textbf{abs}} невозможно вывести из контекста вызова. Поэтому при вызове будет использоваться его значение по умолчанию, содержащее SFINAE-проверку. \\\\
Достоинством этого метода является то, что SFINAE проверка осуществляется на более ранней стадии компиляции --- при \textit{выводе} типов шаблонных параметров.
Это эффективно в смысле временных затрат на компиляцию, по сравнению с использованием \textbf{\texttt{enable\_if}} в сигнатуре 
функции.\\\\
Но у этого метода есть недостаток. Нельзя перегружать функции, отличающиеся лишь SFINAE-проверками, так как значения по умолчанию не входят в сигнатуру функции.
При попытке так сделать произойдет ошибка компиляции, поскольку две функции имеют одинаковые сигнатуры (redefinition):
\begin{minted} [linenos, frame=lines, framesep=2mm, tabsize = 4, breaklines]{c++}
template <typename T,
          typename = typename enable_if<is_signed<T>::value>::type>
T abs(T val) 
{
    return -val;
}

template <typename T,
          typename = typename enable_if<!is_signed<T>::value>::type>
T abs(T val) 
{
    return val;
}
\end{minted}
\begin{minted}[linenos, frame=lines, framesep=2mm, tabsize = 4, breaklines]{text}
error: redefinition of "template<class T, class> T abs(T)"
 T abs(T val)
   ^~~
\end{minted}

\subsection{Constexpr}
\subsubsection{Constant expressions}
\textbf{\textit{Constant expressions}} --- это такие выражения, значения которых должны быть известны на этапе компиляции. Например, требуется, чтобы размер статического массива и значение non-type шаблонного аргумента были \textit{constant expression}'ами.\\
Пример:
\begin{minted} [linenos, frame=lines, framesep=2mm, tabsize = 4, breaklines]{c++}
int n = 1;
std::array<int, n> a1;    // error: n is not a constant expression
int arr[n];               // error: n is not a constant expression
const int cn = 2;
std::array<int, cn> a2;   // OK: cn is a constant expression
int arr[cn];              // OK: cn is a constant expression
\end{minted}

Переменные, помеченные как \textbf{const}, не обязательно являются constant expression'ами.\\ Например:
\begin{minted} [linenos, frame=lines, framesep=2mm, tabsize = 4, breaklines]{c++}
int compute_value()
{
    int result;
    std::cin >> result;
    return result;
}

int const value = compute_value();
int arr[value];         // error: value is not a constant expression

\end{minted}
В данном случае \textbf{const} означает лишь то, что переменную невозможно изменять. Конкретное значение этой переменной известно лишь после запуска программы. \textbf{Const} переменная является \textit{constant expression} только если:
\begin{enumerate}
	\item Она является интегральным типом или перечислением. (см. таблицу \ref{tab:types} категорий типов)
	\item Выражение, которым инициализируется начальное значение --- \textit{constant expression}.
\end{enumerate}
Результат вызова функции в C++03 никогда не является constant expression'ом:
\begin{minted} [linenos, frame=lines, framesep=2mm, tabsize = 4, breaklines]{c++}
int arr[max(3, 4)];         // error: max(3, 4) is not a constant expression
\end{minted}

Введенное в C++11 ключевое слово \textbf{constexpr} решает две задачи:
\begin{enumerate}
\item Для переменных --- позволяет указать, что переменная является \textit{constant expression}.
\item Для функций --- позволяет указать, что результат вызова функции является \textit{constant expression}.
\end{enumerate}
\subsubsection{Constexpr переменные}
Переменные, объявленные как \textbf{constexpr}, должны удовлетворять следующим требованиям:
\begin{enumerate}
\item Должны быть проинициализированы сразу при объявлении.
\item Выражение в инициализации должно быть \textit{constant expression}.
\end{enumerate}
Использование \textbf{constexpr} переменной является \textit{constant expression}'ом. \textbf{Constexpr} переменные неявно являются \textbf{const} переменными. \\\\
Как и \textbf{const} переменные, по умолчанию имеют внутреннюю линковку, в каждой единице трансляции имеют собственный адрес. \\\\
В C++17 появился способ устанавливать внешнюю линковку для \textbf{constexpr} переменных:
\begin{minted} [linenos, frame=lines, framesep=2mm, tabsize = 4, breaklines]{c++}
//a.cpp
struct X {
	static constexpr int x = 42; //declaration
}
\end{minted}
\begin{minted} [linenos, frame=lines, framesep=2mm, tabsize = 4, breaklines]{c++}
//b.cpp
constexpr int X::x; //definition
\end{minted}
Такой код корректен, так как в С++17 все \texttt{static} переменные, являющиеся member'ами какого-то класса, неявно являются inline.
\subsubsection{Constexpr функции}
Объявление функции как \textbf{constexpr} означает, что если все её аргументы являются \textit{constant expression}'ами, то возвращаемое значение может быть посчитано на этапе компиляции. Если значение хотя бы одного аргумента, переданного функции, не известно на момент компиляции, то возвращаемое значение функции вычисляется в run-time, ошибки компиляции не происходит.\\\\
В C++14 \textbf{constexpr} функции могут содержать в теле следующее:
\begin{enumerate}
	\item Любые выражения, кроме \texttt{static} или \texttt{thread\_local} переменных.
	\item Конструкции if и switch.
	\item Циклы.
	\item Выражения, изменяющие значения объектов, если время жизни этих объектов началось в \textbf{constexpr} функции.
\end{enumerate}
Возможна рекурсия. \\\\
По стандарту \textbf{constexpr} функции неявно являются \texttt{inline}. \\\\
Пример:
\begin{minted} [linenos, frame=lines, framesep=2mm, tabsize = 4, breaklines]{c++}
constexpr int sum(int n) {
	int sum = 0;
	for (int i = 1; i <= n; ++i) 
	{
		if (i % 2)
			sum += i;
	}
	return sum;
}

int main()
{
	int a[sum(10)];     // OK: 10 is a constant expression, so sum(10) - constant expression
	int x; 
	std::cin >> x;
	int y = sum(x);
	int b[sum(x)];      // compilation error: x is not a constant expression, so sum(x) is not a constant expression
	return 0;
}
\end{minted}
В строке 13 в \textbf{constexpr} функцию \texttt{sum} передается \textit{constant expression}, поэтому она считается на этапе компиляции. \\\\
В строке 16 передается не \textit{constant expression}. Функция просто считается в run-time. \\\\
Ошибка компиляции в строке 17 подтверждает, что \texttt{sum(x)} не является \textbf{constant expression}.
\subsection{If constexpr}
Пусть требуется написать функцию модуля для всех численных типов. Возможна следующая реализация:
\begin{minted} [linenos, frame=lines, framesep=2mm, tabsize = 4, breaklines]{c++}
template <typename T>
T abs(T value) 
{
	if (is_signed<T>::value) 
		return (value < 0 ? -value : value);
	else 
		return value;
}
\end{minted}
С помощью обычного if-statement'а и метафункции \texttt{\textbf{is\_signed}} тип проверяется на принадлежность к категории знаковых. В зависимости от этой проверки возвращается либо значение переменной, либо ему противоположное. \\\\
Условие в if-statement'е являтся \textit{constant expression}'ом, поэтому уже на этапе компиляции известно, какая ветка if-statement'а будет выполняться всегда, а какая является недостижимой. Несмотря на это, код в недостижимой ветке обычного if-statement'а должен быть корректным в любом случае. \\\\
Это влечет проблемы, если функция \textbf{\texttt{abs}} вызывается от пользовательского числового типа, не имеющего унарного оператора минус (например unsigned\_big\_integer). Подобный вызов приведет к ошибке компиляции. \\\\
Эту проблему решает конструкция \textbf{if constexpr}, появившаяся в C++17:
\begin{minted} [linenos, frame=lines, framesep=2mm, tabsize = 4, breaklines]{c++}
template <typename T>
T abs(T value)
{
	if constexpr(is_signed<T>::value)
		return (value < 0 ? -value : value);
	else
		return value;
}
\end{minted}
В отличии от обычного if-statement'a, \textbf{if constexpr} не требует корректности выражений зависимых от шаблонного параметра в false-ветке.\\\\
Таким образом, \textbf{if constexpr} позволяет создавать compile-time условие. Условие должно являться \textit{constant expression} типа bool.\\\\
Казалось бы, такое мощное средство позволяет полностью избавиться от необходимости 
использования косвенных SFINAE-проверок, использующих шаблоны, и перейти к более удобному синтаксису. Но, в действительности, это не так.
\textbf{If constexpr} не может в полной мере заменить SFINAE. \\\\
При использовании \textbf{if constexpr} compile-time проверка осуществляется на более поздней стадии --- уже внутри тела функции. В случае если перегрузка от данного типа не имеет смысла и не должна быть сгенерирована вовсе, \textbf{if constexpr} все равно позволит коду функции сгенерироваться.\\\\
В привиденном выше примере SFINAE-проверка на наличие метода \textbf{\texttt{abs}} от \textbf{\texttt{std::string}} вернет true, что противоречит желаемой действительности. \\\\
Полностью корректная версия должна содержать SFINAE-проверку:
\begin{minted} [linenos, frame=lines, framesep=2mm, tabsize = 4, breaklines]{c++}
template <typename T>
decltype(enable_if<is_signed<T>::value>::type, T) abs(T value)
{
	return (value < 0 ? -value : value);
}

template <typename T>
decltype(enable_if<is_unsigned<T>::value>::type, T) abs(T value)
{
	return value;
}
\end{minted}
\subsection{Static assertions}
Предоставляет возможность писать compile-time asserts. \\\\
Синтаксис: \\
\texttt{static\_assert(condition, message)} \\\\
где \texttt{condition} --- это \textit{constant expression} типа bool, \texttt{message} --- строковый литерал, сообщение выдаваемое при срабатывании ассерта. \\\\
В случае condition $=$ false, происходит ошибка компиляции, печатается сообщение message, иначе ничего не происходит. \\\\
Применяется для compile-time проверки свойств, зависящих от параметра шаблона:
\begin{minted} [linenos, frame=lines, framesep=2mm, tabsize = 4, breaklines]{c++}
template<class Integral>
Integral foo(Integral x, Integral y) {
	static_assert(std::is_integral<Integral>::value, "foo() parameter must be an integral type.");
}
\end{minted}
\subsection{Template variables}
В С++14 появилась возможность создавать шаблонные переменные.
\begin{minted} [linenos, frame=lines, framesep=2mm, tabsize = 4, breaklines]{c++}
template <typename T>
constexpr T pi = T(3.14159265);

constexpr float x = pi<double>;
constexpr float y = pi<float>;
\end{minted}
Шаблонные переменные обязаны быть \textbf{constexpr}. Допускаются специализации. \\\\
Наиболее часто применяются для упрощения метафункций из \texttt{std::type\_traits}. \\
Рассмотрим на примере \texttt{is\_same}:
\begin{minted} [linenos, frame=lines, framesep=2mm, tabsize = 4, breaklines]{c++}
template <typename T, typename U>
constexpr bool is_same_v = is_same<U, V>::value;
\end{minted}
Теперь вместо \texttt{is\_same<T, U>::value} можем писать просто \texttt{is\_same\_v<T, U>}. \\\\
В C++14 все метафункции, возвращающие значение, имеют соответствующие шаблонные переменные с суффиксом \texttt{\_v}.
\subsection{Type alias \& Alias template}
\textbf{Type alias} --- это псевдоним типа, объявленного ранее(подобно \texttt{typedef}). \textbf{Template alias} --- псевдоним для шаблона.\\\\
\textbf{Type alias} можно использовать, как альтернативу \texttt{typedef}: \\\\
\texttt{using mytype = int;} \\\\
\textbf{Alias template} позволяет написать собственный псевдоним с шаблонным параметром, например: \\
Синтаксис:
\begin{minted} [frame=lines, framesep=2mm, tabsize = 4, breaklines]{c++}
template <typename T>
using my_vector = vector<T, mmap_allocator<T>>;

my_vector<int> a; //vector<int, mmap_allocator<int>> a;
\end{minted}
Как и \textbf{template variables}, \textbf{alias templates} используются для упрощения \texttt{std::type\_traits}. \\\\
В C++14 все метафункции, возвращающие тип, имеют соответствующий \textbf{alias template} с суффиксом \texttt{\_t} (\texttt{enable\_if\_t<T, U>}).\\
Позволяют сильно упростить код, так как пропадает необходимость писать \texttt{typename} и \texttt{::type}.
\subsection{Явное инстанцирование шаблонов.}
Явное инстанцирование шаблона позволяет сгенерировать код шаблонной функции или класса для конкретных параметров без использования его в коде. \\\\
Синтаксис:
\begin{minted} [frame=lines, framesep=2mm, tabsize = 4, breaklines]{c++}
template <typename T> 
class my_class {}

template <typename T, typename U, typename V>
V f(T, U) {}

template class my_class<int>; // for classes
template double f(int, char); // for functions
\end{minted}
При таком инстанцировании инстанцируются все не-template мемберы.
\subsection{Подавление инстанцирования шаблонов.}
В C++11 появилась возможность подавления неявного инстанцирования шаблона. \\\\
Синтаксис:
\begin{minted} [frame=lines, framesep=2mm, tabsize = 4, breaklines]{c++}
extern template class my_class<int>; // for classes
extern template double f(int, char); // for functions
\end{minted}
Можно использовать это для уменьшения количества генерируемого кода. Например, можно подавить инстанцирование в header-файле и явно проинстанцировать в одном cpp-файле. \\
Так оптимизируется время компиляции.
\section{std::type\_traits}
\texttt{std::type\_traits} --- это библиотека, которая содержит множество полезных метафункции для работы с классами. В этой части будет проведен её обзор. \\\\
\subsection{Вспомогательные типы}
\begin{itemize}
	\item \texttt{\textbf{integral\_constant<class T, T v}>} \\ Константа времени компиляции типа T со значением v. \\
	Содержит поле: \\
\texttt{\textbf{static constexpr T value = v;}}
	\item \texttt{\textbf{true\_type}} определен как integral\_constant<bool, true>
	\item \texttt{\textbf{false\_type}} определен как integral\_constant<bool, false>
\end{itemize}
Довольно удобно наследоваться от последних двух при написании своих метафункций:
\begin{minted} [frame=lines, framesep=2mm, tabsize = 4, breaklines]{c++}
template<class T, class U>
struct is_same : std::false_type {};

template<class T>
struct is_same<T, T> : std::true_type {};
\end{minted}
\subsection{Проверка на принадлежность типа к определенной категории}
\begin{table}[H]
	\caption{\label{tab:types} Категории типов}
	\begin{center}
\begin{tabular}{|c|c|c|c|c|}
	\hline 
	& \textbf{primary categories} & \multicolumn{3}{|c|}{\textbf{composite categories}} \\ 
	\hline 
	\multirow{4}{*}{\textbf{fundamental}} & void &  &  &  \\ 
	\cline{2-5}
	& std::nullptr\_t &  & \multirow{7}{*}{scalar} & \multirow{10}{*}{object} \\ 
	\cline{2-3} 
	& integral & \multirow{2}{*}{arithmetic} &  & \\ 
	\cline{2-2} 
	& floating\_point  &  &  &  \\ 
	\cline{1-3} 
	\multirow{10}{*}{\textbf{compound}}& pointer &  &  &  \\ 
	\cline{2-3} 
	& member\_object\_pointer  & \multirow{2}{*}{member\_pointer}  &  &  \\ 
	\cline{2-2} 
	& member\_function\_pointer  &  &  &  \\ 
	\cline{2-3} 
	& enum &  &  &  \\ 
	\cline{2-4} 
	& union &  &  &  \\ 
	\cline{2-4} 
	& class* &  &  &  \\ 
	\cline{2-4} 
	& array &  &  &  \\ 
	\cline{2-5} 
	& l\_value\_reference  & \multirow{2}{*}{reference} &  &  \\ 
	\cline{2-2} \cline{4-5} 
	& r\_value\_reference &  &  &  \\ 
	\cline{2-5} 
	& function  &  &  &  \\ 
	\cline{1-5} 
	\multicolumn{5}{|c|}{* = excluding unions} \\ 
	\hline 
\end{tabular}
\end{center}
\end{table}
К интегральному типу относятся следующе встроенные типы:\\\\
\texttt{bool, char, char16\_t, char32\_t, wchar\_t, short, int, long, long long}. \\\\
\textbf{type\_traits} предоставляет набор метафункций, позволяющих проверить, принадлежит ли тип T к какой-либо категории из приведенных в таблице выше. \\\\
Название таких метафункций имеет вид: \texttt{is\_/*название\_категории*/}.
\subsection{Проверка типа на наличие определенных свойств}
В C++11 поддерживается широкий набор метафункций для проверки наличия у типа определенных свойств. Рассмотрим основные из них:
\begin{itemize}
\item \textbf{\texttt{is\_const}} \\\\
Проверяет наличие спецификатора const или const volaitile.
\item \textbf{\texttt{is\_volatile}} \\\\
Проверяет наличие спецификатора volaitile или const volaitile.
\item \textbf{\texttt{is\_signed}} \\\\
Проверяет, что если тип является арифметическим, то выражение T(-1) < T(0) = true.
То есть тип --- знаковый.
\item \textbf{\texttt{is\_unsigned}} \\\\
Проверяет, что если тип является арифметическим, то выражение T(0) < T(-1) = true.
То есть тип --- беззнаковый.
\item \texttt{\textbf{is\_trivially\_copyable}} \\\\
Проверяет, что тип --- \textit{тривиально копируемый}. \\\\
Тип считается \textit{тривиально копируемым}, если:
\begin{enumerate} 
\item Любой его конструктор копирования, move конструктор, копирующий оператор присваивания, 
move оператор присваивания является \textit{тривиальным} или помечен как deleted.
\item Хотя бы один из перечисленных не помечен как deleted.
\item Имеет \textit{тривиальный} деструктор, не помеченный как deleted.
\end{enumerate}
Все типы, совместимые с C, тривиально копируемы. \\\\
Конструктор копирования, move конструктор, копирующий оператор присваивания, 
move оператор присваивания (далее --- операция) в классе T называется \textit{тривиальным}, если:
\begin{enumerate} 
\item Он не определен пользователем (определен неявно, или помечен как default).
\item T не имеет виртуальных функций.
\item T не имеет виртуальных баз.
\item Соответствующая операция, выбирающаяся для прямой базы T, тривиальна.
\item Соответствующая операция, выбирающаяся для каждого не статического класса-мембера, тривиальна.
\end{enumerate}
Конструктор/деструктор называется \textbf{\textit{тривиальным}}, если:
\begin{enumerate}
	\item Он не определен пользователем (определен неявно или помечен как default).
	\item Он не виртуальный.
	\item Все прямые базы класса имеют тривиальный конструктор/деструктор.
	\item Все не статические мемберы класса имеют тривиальный конструктор/деструктор.
\end{enumerate}
За счет наложенных ограничений тривиально копируемый тип может копироваться побайтово. Например, с помощью \texttt{std::memcpy}.
\item \textbf{\texttt{is\_trivial}} \\\\
Тип считается тривиальным, если он:
\begin{enumerate}
\item Тривиально копируемый.
\item Имеет хотя бы 1 конструктор по умолчанию, все они \textbf{\textit{тривиальны}} или помечены как deleted,
и хотя бы 1 из них не помечен как deleted.
\end{enumerate}
\item \texttt{\textbf{is\_constructible}} $(1)$\\
		\texttt{\textbf{is\_trivially\_constructible}} $(2)$\\
		\texttt{\textbf{is\_nothrow\_constructible}} $(3)$\\\\
Имеет аргументы: template<class T, class... Args>
\begin{enumerate}
\item Проверяет, что выражение \texttt{T obj(std::declval<Args>()}, вызывающее констурктор от типа корректно.
\item Проверяет (1) и что вызов конструктора не вызывает \textit{не тривиальных} операций.
\item Проверяет (1) и что вызов конструктора noexcept.
\end{enumerate}
Есть аналогичные метафункции, проверяющие \texttt{default\_constructible, copy\_constructible, move\_constructible}. Они выражаются через соответствующие версии \texttt{is\_constructible}.
\textbf{\texttt{\item is\_assignable}} $(1)$ \\
\textbf{\texttt{is\_trivially\_assignable}} $(2)$ \\
\textbf{\texttt{is\_nothrow\_assignable}} $(3)$ \\\\
Имеет аргументы: template<class T, class U> 
\begin{enumerate}
\item Проверяет, что выражение \texttt{std::declval<T>() = std::declval<U>()}, вызывающее оператор присваивания корректно.
\item Проверяет (1) и что вызов конструктора не вызывает \textit{не тривиальных} операций.
\item Проверяет (1) и что вызов конструктора noexcept.
\end{enumerate}
Есть аналогичные метафункции, проверяющие \texttt{copy\_assignable, move\_assignable}. Они выражаются через соответствующие версии \texttt{is\_assignable}. \\\\
\item Абсолютно аналогично строится тройка метафункций 
\texttt{\textbf{is\_destructible, is\_trivially\_destructible, is\_nothrow\_destructible}}
и пара \texttt{\textbf{is\_swapable, is\_nothrow\_swapable}}.
\end{itemize}
\subsection{Взаимоотношения типов}
\begin{itemize}
	\item \textbf{\texttt{is\_same<T, U>}} --- проверяет, что типы T и U совпадают.
	\item \textbf{\texttt{is\_base\_of<Base, Derived>}} --- проверяет, что Derived наследуется от Base или они совпадают.
	\item \textbf{\texttt{is\_convertible<From, To>}} --- проверяет корректность следующего определения функции: 
	\\\\ \texttt{To test() \{ return std::declval<From>(); \}}\\\\
	что обеспечивает возможность неявного конвертирования From в To.
\end{itemize}
\subsection{Трансформирование типов}
Также в \textbf{type\_traits} определены средства трансформации типов, имеющие интуитивно понятные названия:
\begin{itemize}
\item 	\texttt{\textbf{remove\_cv}}
\item	\texttt{\textbf{remove\_const}}
\item	\texttt{\textbf{remove\_volatile}}
\item	\texttt{\textbf{add\_cv}}
\item	\texttt{\textbf{add\_const}}
\item	\texttt{\textbf{add\_volatile}}
\item	\texttt{\textbf{remove\_reference}}
\item	\texttt{\textbf{add\_lvalue\_reference}}
\item	\texttt{\textbf{add\_rvalue\_reference}}
\item	\texttt{\textbf{remove\_pointer}}
\item	\texttt{\textbf{add\_pointer}}
\end{itemize}