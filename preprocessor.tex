\section{Препроцессирование}

Препроцессирование это первая стадия компиляции. Препроцессирование выполняется программой называемой препроцессор. Пропроцессор получает на вход файл исходного кода с расширением .cpp. Вывод препроцессора передается на следующую стадию компиляции --- транслятору.

Препроцессор проходит по входному файлу, копируя текст в выходной файл. Если он встречает строку начинающуются с символа {\bf \#}, он выполняет команду, которая записана после {\bf \#}. Такие команды называются директивами препроцессора.

\subsection{Директивы препроцессора}

Существуют следующие директивы препроцессора:
\begin{enumerate}
\item \#define
\item \#elif
\item \#else
\item \#error
\item \#endif
\item \#if
\item \#ifdef
\item \#ifndef
\item \#include
\item \#pragma
\item \#undef
\end{enumerate}

\subsubsection{Директива \#include}

Директива {\bf \#include} имеет две формы
\begin{minted}[linenos, frame=lines, framesep=2mm, tabsize = 4, breaklines]{c++}
#include <somefile.h>
#include "somefile.h"
\end{minted}

Когда препроцесор встречает эту директиву, он подставляет на её место текст файла {\bf somefile.h}. При этом текст файла {\bf somefile.h} тоже препроцессируется. Если в нём есть какие-то директивы то они будут выполнены.

Пусть даны следующие файлы
\begin{minted}[linenos, frame=lines, framesep=2mm, tabsize = 4, breaklines]{c++}
// print.h
void print(char const*);

// main.cpp
#include "print.h"

int main()
{
    print("Hello, world");
}
\end{minted}
Результатом препроцессирования {\bf main.cpp} будет:
\begin{minted}[linenos, frame=lines, framesep=2mm, tabsize = 4, breaklines]{c++}
void print(char const*);

int main()
{
    print("Hello, world");
}
\end{minted}
В этом можно убедиться --- выполним команду {\bf g++ -E main.cpp}. Ключ {\bf -E} говорит драйверу, что нужно лишь отпрепроцессировать входной файл, не выполняя дальнейшие стадии компиляции.

Форма {\bf \#include <somefile.h>} производит поиск файла {\bf somefile.h} в списке каталогов, называемых include directories. Их можно указать компилятору опцией {\bf -I}. Например:
\begin{minted}[linenos, frame=lines, framesep=2mm, tabsize = 4, breaklines]{sh}
g++ -Isome/directory -Ianother/directory -I../and/another/one myfile.cpp
\end{minted}
Файлы ищутся в этих директориях слева направо. Препроцессор автоматически добавляет в конец этого списка пути к директориям в которых лежат хедера стандартной библиотеки.

Форма {\bf \#include ``somefile.h''} производит поиск файла {\bf somefile.h} сначала в текущем каталоге, а потом в include directories.

\subsubsection{Директива \#define}

Рассмотрим директиву {\bf \#define} на примере:
\begin{minted}[linenos, frame=lines, framesep=2mm, tabsize = 4, breaklines]{c++}
#define PI 3.14159265

double circumference(double r)
{
    return 2 * PI * r;
}
\end{minted}
Здесь директива {\bf \#define} определяет макрос с именем {\bf PI}. Текст, который идет после имени макроса, называется replacement'ом. Replacement отделяется от имени макроса пробелом и распростроняется до конца строки. Все вхождения идентификатора {\bf PI} ниже этой директивы будут заменены на replacement. Результатом препроцессирования примера выше является следующий текст:
\begin{minted}[linenos, frame=lines, framesep=2mm, tabsize = 4, breaklines]{c++}
double circumference(double r)
{
    return 2 * 3.14159265 * r;
}
\end{minted}

Приведенная выше форма директивы {\bf \#define} называется object-like. Существует вторая форма этой директивы, называемая function-like:
\begin{minted}[linenos, frame=lines, framesep=2mm, tabsize = 4, breaklines]{c++}
#define MIN(x, y) x < y ? x : y

printf("%d", MIN(4, 5));
\end{minted}
Результатом препроцессирования этого фрагмента кода является:
\begin{minted}[linenos, frame=lines, framesep=2mm, tabsize = 4, breaklines]{c++}
printf("%d", 4 < 5 ? 4 : 5);
\end{minted}

Важно знать, что препроцессор, выполняя подстановки макросов, ничего не знает про приоритет арифметических операций и синтаксическую структуру программы. Рассмотрим следующую программу:
\begin{minted}[linenos, frame=lines, framesep=2mm, tabsize = 4, breaklines]{c++}
#define MIN(x, y) x < y ? x : y

int main()
{
    printf("%d", 10 + MIN(4, 5));
}
\end{minted}
Данная программа выводит {\bf 5}, тогда как скорее всего программист ожидал вывода {\bf 14}. Дело в том, что после раскрытия макроса возникает выражение {\bf 10 + 4 < 5 ? 4 : 5}. Поскольку бинарный {\bf +} имеет приоритет выше, чем у тернарного оператора, данное выражение разбирается транслятором как {\bf (10 + 4) < 5 ? 4 : 5}, а не {\bf 10 + (4 < 5 ? 4 : 5)}, как мог ожидать программист, использующий макрос. Чтобы избегать подобных проблем, у function-like макросов, которые раскрываются в выражение, replacement следует брать в скобки. По той же причине имена параметров макроса в replacement, следует брать в скобки. Корректный макрос {\bf MIN} мог бы выглядеть следующим образом:
\begin{minted}[linenos, frame=lines, framesep=2mm, tabsize = 4, breaklines]{c++}
#define MIN(x, y) ((x) < (y) ? (x) : (y))
\end{minted}

Директива {\bf define} позволяет определять макросы повторно, при этом, в каждой точке программы силу имеет последний {\bf define} данного макроса:
\begin{minted}[linenos, frame=lines, framesep=2mm, tabsize = 4, breaklines]{c++}
X
#define X foo
X
#define X bar
X
\end{minted}
раскрывается в:
\begin{minted}[linenos, frame=lines, framesep=2mm, tabsize = 4, breaklines]{c++}
X
foo
bar
\end{minted}

Replacement макроса не препроцессируется при определении макроса, но результат раскрытия макроса препроцессируется повторно:
\begin{minted}[linenos, frame=lines, framesep=2mm, tabsize = 4, breaklines]{c++}
#define Y foo
#define X Y
#define Y bar
X                   // раскрывается в bar
\end{minted}

Что произойдет если replacement макроса {\bf M} будет содержать использование макроса {\bf M}? В этом случае возникает рекурсия. По спецификации препроцессор никогда не должен раскрывать макрос {\bf M} изнутри самого себя, а оставлять вложенный идентификатор как есть:

\begin{minted}[linenos, frame=lines, framesep=2mm, tabsize = 4, breaklines]{c++}
#define M { M }
M                   // раскрывается в { M } один раз, второй раз M не раскрывается
\end{minted}

Ещё пример:
\begin{minted}[linenos, frame=lines, framesep=2mm, tabsize = 4, breaklines]{c++}
#define A a{ B }
#define B b{ C }
#define C c{ A }
A
B
C
\end{minted}
Результат препроцессирования:
\begin{minted}[linenos, frame=lines, framesep=2mm, tabsize = 4, breaklines]{c++}
a{ b{ c{ A } } }
b{ c{ a{ B } } }
c{ a{ b{ C } } }
\end{minted}

\subsubsection{Директива \#undef}

Директива {\bf \#undef} позволяет разопределить макрос, определенный ранее с помощью директивы {\bf \#define}. Пример:
\begin{minted}[linenos, frame=lines, framesep=2mm, tabsize = 4, breaklines]{c++}
X
#define X foo
X
#undef X
X
\end{minted}
Результат препроцессирования:
\begin{minted}[linenos, frame=lines, framesep=2mm, tabsize = 4, breaklines]{c++}
X
foo
X
\end{minted}

\subsubsection{Семейство директив \#if}

Директивы {\bf \#ifdef}, {\bf \#ifndef}, {\bf \#if}, {\bf \#else}, {\bf \#elif}, {\bf \#endif} позволяют отпрепроцессировать часть файла, лишь при определенном условии. Директивы {\bf \#ifdef}, {\bf \#ifndef} проверяют определен ли указанный макрос. Например:
\begin{minted}[linenos, frame=lines, framesep=2mm, tabsize = 4, breaklines]{c++}
#ifdef __x64_64__
typedef unsigned long uint64_t;
#else
typedef unsigned long long uint64_t;
#endif
\end{minted}

Директива \#if позволяет проверить произвольное арифметическое выражение.
\begin{minted}[linenos, frame=lines, framesep=2mm, tabsize = 4, breaklines]{c++}
#define TWO 2
#if TWO + TWO == 4
// ...
#endif
\end{minted}
Директива {\bf \#if} препроцессирует свой аргумент, а затем парсит то, что получилось как арифметическое выражение. Если после препроцессирования в аргументе {\bf \#if} остаются идентификаторы, то они заменяются на 0, кроме идентификатора {\bf true}, который заменяется на 1.

\subsubsection{Директива \#error}

Директива {\bf \#error} обрывает препроцессирование с ошибкой. Ей можно указать сообщение которое будет показано в ошибке. Например:
\begin{minted}[linenos, frame=lines, framesep=2mm, tabsize = 4, breaklines]{c++}
#if __BYTE_ORDER == __BIG_ENDIAN
// ...
#elif __BYTE_ORDER == __LITTLE_ENDIAN
// ...
#else
#error unknown endianness
#endif
\end{minted}

\subsubsection{Директива \#pragma}

Директива {\bf \#pragma} это директива поведение, которой может быть определено каждым компилятором по своему (implementation-defined behavior). То, какие формы директивы {\bf \#pragma} поддерживает конкретный компилятор, можно узнать в документации. Например, компилятор Microsoft Visual C++ поддерживает 41\footnote{\url{https://msdn.microsoft.com/en-us/library/d9x1s805.aspx}} вид разных прагм.

Наиболее широко используемой прагмой является {\bf \#pragma once}, рассмотренная в разделе \ref{pragma_once}.

\subsection{Защита от повторного включения}

При препроцессировании директива {\bf \#include} заменяется на текст указаного хедер-файла. Повторное включение хедер файл может привести к тому, что декларации во втором включении конфликтуют с декларациями в первом включении. Рассмотрим следующую программу:
\begin{minted}[linenos, frame=lines, framesep=2mm, tabsize = 4, breaklines]{c++}
// mytype.h
struct mytype
{
    int abc;
};

// somefile.cpp
#include "mytype.h"
#include "mytype.h"
\end{minted}

Препроцессирование {\bf somefile.cpp} выдаёт следующий результат:
\begin{minted}[linenos, frame=lines, framesep=2mm, tabsize = 4, breaklines]{c++}
struct mytype
{
    int abc;
};

struct mytype
{
    int abc;
};
\end{minted}

Компиляция {\bf somefile.cpp} ожидаемо приводит к ошибке:

\begin{minted}[linenos, frame=lines, framesep=2mm, tabsize = 4, breaklines]{text}
In file included from somefile.cpp:2:0:
mytype.h:1:8: error: redefinition of ‘struct mytype’
 struct mytype
        ^~~~~~
In file included from somefile.cpp:1:0:
mytype.h:1:8: error: previous definition of ‘struct mytype’
 struct mytype
        ^~~~~~
\end{minted}

Как видно из ошибки структура {\bf mytype} сконфликтовала с собой же.

В данной ситуации можно было бы посоветовать не инклудить хедер дважды. Однако это не всегда легко сделать. Предположим, что существует два хедера {\bf a.h} и {\bf b.h}, которые используют {\bf mytype} из {\bf mytype.h}, а файл {\bf somefile.cpp} использует оба хедера {\bf a.h} и {\bf b.h}. Следует ли в {\bf a.h} и в {\bf b.h} инклудить {\bf mytype.h}? Если ответить на этот вопрос положительно, то это приведет к аналогичной ошибке, за тем лишь исключением, что {\bf mytype.h} инклудиться не напрямую в {\bf somefile.cpp}, а косвенно --- первый раз через {\bf a.h}, второй --- через {\bf b.h}. Если ответить на этот вопрос отрицательно, то каждый пользователь файлов {\bf a.h} будет обязан перед инклудом {\bf a.h} проинклудить {\bf mytype.h}. Для одного хедер-файла это может показаться приемлемым, однако, если программа состоит из сотен хедеров, каждый требующий десятки других хедеров, этот подход становится неприемлемым.

Решением данной проблемы является написание хедера таким образом, что его возможно включать несколько раз и это не приводит к ошибке. Один возможный способ это сделать показан на следующем примере:

\begin{minted}[linenos, frame=lines, framesep=2mm, tabsize = 4, breaklines]{c++}
// mytype.h
#ifndef MYTYPE_H
#define MYTYPE_H

struct mytype
{
    int abc;
};

#endif
\end{minted}

Такая тройка директив {\bf \#ifndef}, {\bf \#define}, {\bf \#endif} называется {\bf include guard'ом}. При включении файла первый раз макрос {\bf MYTYPE\_H} неопределен, поэтому текст файла будет подставлен, а макрос {\bf MYTYPE\_H} станет определен. При повторном включении макрос {\bf MYTYPE\_H} уже определен и содержимое файла пропускается.

При использовании include guard'ов важно, чтобы макрос, который проверяется и определяется был уникальный для каждого файла. Обычно для этого используют имя макроса образованное от имени файла.

\subsubsection{\#pragma once}
\label{pragma_once}
Вторым способом защитить файл от повторного включения является использование директивы {\bf \#pragma once}:

\begin{minted}[linenos, frame=lines, framesep=2mm, tabsize = 4, breaklines]{c++}
// mytype.h
#pragma once

struct mytype
{
    int abc;
};

// somefile.cpp
#include "mytype.h"
#include "mytype.h"
\end{minted}

Директива {\bf \#pragma once} говорит препроцессору, что текст файла следует подставлять лишь при первом включение. Препроцессирование {\bf somefile.cpp} из примера выше выдаёт:

\begin{minted}[linenos, frame=lines, framesep=2mm, tabsize = 4, breaklines]{c++}
struct mytype
{
    int abc;
};
\end{minted}

Преимуществом использования {\bf \#pragma once} по сравнению с include guard является то, что не требуется изобретать новое имя макроса для каждого нового файла. Это исключает ошибку в случае если макросы совпадут для разных файлов.

Недостатком использования {\bf \#pragma once} является то, что она не входит в стандарт C++, а является лишь расширением, пусть и очень популярным, компиляторов. Эта директива поддерживается очень широко, поэтому её использовании маловероятно приведет к проблемам с переносимостью на практике. GCC, clang, Microsoft Visual C++, Intel C++ все поддерживают эту директиву.

\subsubsection{Рекурсивное включение}
\label{recursive_include}

Иногда, при не очень аккуратном программировании возникает ситуация, когда хедер {\bf a.h} инклудит {\bf b.h}, а хедер {\bf b.h} инклудит {\bf a.h}. При отсутствии инклуд-guard'ов это привело бы к ошибке компиляции:

\begin{minted}[linenos, frame=lines, framesep=2mm, tabsize = 4, breaklines]{c++}
// a.h
#include "b.h"

// b.h
#include "a.h"

// c.cpp
#include "a.h"
\end{minted}

\begin{minted}[linenos, frame=lines, framesep=2mm, tabsize = 4, breaklines]{text}
In file included from b.h:1:0,
                 from a.h:1,
                 from b.h:1,
                 from a.h:1,
                 from b.h:1,
< ... пропущено ... >
                 from a.h:1,
                 from b.h:1,
                 from a.h:1,
                 from b.h:1,
                 from a.h:1,
                 from c.cpp:1:
a.h:1:15: error: #include nested too deeply
 #include "b.h"
               ^
\end{minted}

При наличии инклуд-guard'ы в хедерах {\bf a.h} и {\bf b.h} проявления ошибок являются более интересными. Рассмотрим на примере:
\begin{minted}[linenos, frame=lines, framesep=2mm, tabsize = 4, breaklines]{c++}
// a.h
#ifndef A_H
#define A_H
#include "b.h"

struct a
{
    b* x;
};
#endif

// b.h
#ifndef B_H
#define B_H
#include "a.h"

struct b
{
    a* x;
};
#endif

// c.cpp
#include "a.h"
\end{minted}

Компиляция {\bf c.cpp} приводит к ошибке:
\begin{minted}[linenos, frame=lines, framesep=2mm, tabsize = 4, breaklines]{text}
In file included from a.h:3:0,
                 from c.cpp:1:
b.h:7:5: error: ‘a’ does not name a type
     a* x;
     ^
\end{minted}

Если же {\bf c.cpp} включает {\bf b.h} ошибка звучит:
\begin{minted}[linenos, frame=lines, framesep=2mm, tabsize = 4, breaklines]{text}
In file included from b.h:3:0,
                 from c.cpp:1:
a.h:7:5: error: ‘b’ does not name a type
     b* x;
     ^
\end{minted}

Чтобы понять почему возникает подобная ошибка следует посмотреть на результат препроцессирования. В первом случае он такой:
\begin{minted}[linenos, frame=lines, framesep=2mm, tabsize = 4, breaklines]{text}
struct b
{
    a* x;
};
struct a
{
    b* x;
};
\end{minted}

Во втором такой:
\begin{minted}[linenos, frame=lines, framesep=2mm, tabsize = 4, breaklines]{text}
struct a
{
    b* x;
};
struct b
{
    a* x;
};
\end{minted}

Действительно, какую-бы структуру не объявляли первой, она не сможет сослаться на вторую поскольку вторая ещё не объявлена. В данном примере эти два хедера вообще не должны друг друга инклудить, поскольку достаточно объявления forward-деклараций:

\begin{minted}[linenos, frame=lines, framesep=2mm, tabsize = 4, breaklines]{c++}
// a.h
#ifndef A_H
#define A_H
struct b;

struct a
{
    b* x;
};
#endif

// b.h
#ifndef B_H
#define B_H
struct a;

struct b
{
    a* x;
};
#endif
\end{minted}

В данном примере какой бы из рекурсивно инклудящих друг-друга хедеров не проинклудить снаружи, мы получаем ошибку. В реальном коде встречаются рекурсивные инклуды, которые приводят к ошибке только если проиклудить один из них снаружи. Так везёт, что снаружи всегда инклудят какой-то один. Тем не менее это ошибка и рекурсивное включение следует избегать. В тех местах, где оно встретилось, как правило, от него можно избавиться использованием forward-деклараций, а так же разбитием большого хедера на части, с более аккуратным контролем зависимостей между частями.

\subsubsection{Защита от рекурсивного включения}

Возможно написать такой include-guard, который будет распозновать рекурсивное включение и выдавать ошибку об этом. Рассмотрим пример с рекурсивным включением из раздела \ref{recursive_include}, но в качестве include-guard будем использовать такой, который детектит рекурсивное включение:

\begin{minted}[linenos, frame=lines, framesep=2mm, tabsize = 4, breaklines]{c++}
// a.h
#ifndef A_H
#define A_H
#define INSIDE_A_H
#include "b.h"

struct a
{
    b* x;
};

#undef INSIDE_A_H
#else
#ifdef INSIDE_A_H
#error recursive inclusion
#endif
#endif

// b.h
#ifndef B_H
#define B_H
#define INSIDE_B_H
#include "a.h"

struct b
{
    a* x;
};

#undef INSIDE_B_H
#else
#ifdef INSIDE_B_H
#error recursive inclusion
#endif
#endif

// c.cpp
#include "a.h"
\end{minted}

Препроцессирование такого {\bf c.cpp} приведет к следующей ошибке:
\begin{minted}[linenos, frame=lines, framesep=2mm, tabsize = 4, breaklines]{text}
In file included from b.h:4:0,
                 from a.h:4,
                 from c.cpp:1:
a.h:14:2: error: #error recursive inclusion
 #error recursive inclusion
  ^~~~~
\end{minted}

Как видно изнутри {\bf a.h} инклудится {\bf b.h}, который инклудит {\bf a.h}, который выдаёт ошибку, что он заинклужен рекурсивно.

Другим способом отловить рекуривное включение является опредение макроса в include-guard не до хедеров а после:

\begin{minted}[linenos, frame=lines, framesep=2mm, tabsize = 4, breaklines]{c++}
// a.h
#ifndef A_H

#include "b.h"

struct a
{
    b* x;
};

#define A_H
#endif

// b.h
#ifndef B_H

#include "a.h"

struct b
{
    a* x;
};

#define B_H
#endif

// c.cpp
#include "a.h"
\end{minted}

\subsection{Проблемы при использовании препроцессора}
Основной сложностью\footnote{\url{http://www.stroustrup.com/bs_faq2.html\#macro} --- см. также FAQ Бьярна Страуструпа на тему, почему макросы это плохо.} при использовании макросов препроцессора является то, что препроцессор оперирует на уровне токенов, не зная ничего про контекст где макрос раскрывается. Предположим, что определен макрос {\bf errno}, а где-то ниже программист пытается определить локальную переменную {\bf errno}.
\begin{minted}[linenos, frame=lines, framesep=2mm, tabsize = 4, breaklines]{c++}
#define errno (*errno_location())

int process()
{
    int errno = 0;
}
\end{minted}
Результатом препроцессирования этого фрагмента будет:
\begin{minted}[linenos, frame=lines, framesep=2mm, tabsize = 4, breaklines]{c++}
int process()
{
    int (*errno_location()) = 0;
}
\end{minted}
Как мы видим получившийся в результате препроцессирования фрагмент не объявляет переменную {\bf errno}. Этот фрагмент не объявляет вообще никакую переменную. В данном конкретном случае программист получит ошибку трансляции:
\begin{minted}[linenos, frame=lines, framesep=2mm, tabsize = 4, breaklines]{text}
errno.cpp:5:17: error: function ‘int* errno_location()’ is initialized like a variable
     int errno = 0;
                 ^
\end{minted}
Сообщение об ошибке ссылается на некоторую функцию {\bf int* errno\_location()}, которая в пользовательском коде не упоминается. Такие ошибки могут быть запутывающими. При использовании библиотек, которые определяют много макросов с короткими именами это может доставлять неудобство, поскольку эти имена становиться невозможно использовать ни под что другое.

Второй проблемой при использовании пропроцессора является то, что отладчик ничего не знает про раскрытые макросы. Независимо от того, сколько кода пришло из макроса, отладчик будет работать как будто весь этот код написан в одной строке. Поэтому, как правило, в коде активно использующем макросы, сложно работать с отладчиком.

\subsection{Советы и лучшие практики}
\subsubsection{Предпочитайте альтернативы}

В связи с расширение языка, сейчас возможно использовать обычные языковые конструкции, там где раньше использовался препроцессор. Например, изначально препроцессор использовался для того, чтобы определять константы:

\begin{minted}[linenos, frame=lines, framesep=2mm, tabsize = 4, breaklines]{c++}
#define BUFF_SIZE 10240
\end{minted}

Без использования препроцессора эту константу возможно объявить как:
\begin{minted}[linenos, frame=lines, framesep=2mm, tabsize = 4, breaklines]{c++}
size_t const BUFF_SIZE = 10240;
\end{minted}

Аналогично function-like макросы часто можно заменить на inline-функции:
\begin{minted}[linenos, frame=lines, framesep=2mm, tabsize = 4, breaklines]{c++}
#define STREQ(s1, s2) (strcmp((s1), (s2)) == 0)
\end{minted}
\begin{minted}[linenos, frame=lines, framesep=2mm, tabsize = 4, breaklines]{c++}
bool streq(char const* s1, char const* s2)
{
    return strcmp(s1, s2) == 0;
}
\end{minted}

В случае когда типы аргументов могут быть различными возможно использование шаблонов:
\begin{minted}[linenos, frame=lines, framesep=2mm, tabsize = 4, breaklines]{c++}
#define MIN(x, y) ((x) < (y) ? (x) : (y))
\end{minted}
\begin{minted}[linenos, frame=lines, framesep=2mm, tabsize = 4, breaklines]{c++}
template <class T>
T const& min(T const& x, T const& y)
{
    return x < y ? x : y;
}
\end{minted}

\subsubsection{Своевременно undef-айнте макросы}

Если макрос предназначен лишь для использования внутри одной функции не забудьте его undef-айнуть по завершению этой функции:

\begin{minted}[linenos, frame=lines, framesep=2mm, tabsize = 4, breaklines]{c++}
enum color
{
    red,
    green,
    blue,
    // ...
};

char const* color_name(color c)
{
    switch (c)
    {
#define CASE(name) case name: return #name
        CASE(red);
        CASE(green);
        CASE(blue);
        // ...
#undef CASE
    }
}
\end{minted}

\subsubsection{Избегайте коротких имен}

Если необходимо объявить глобально доступный макрос, подберите ему такое имя, которое маловероятно приведет к колизии. {\bf GTK\_MAJOR\_VERSION} --- пример хорошего имени макроса. {\bf min}, {\bf check}, {\bf tmp}, {\bf out} --- примеры плохих имен.
