\documentclass[12pt]{article}

\usepackage[utf8]{inputenc} % common
\usepackage[russian]{babel} % for russian lang

\usepackage{graphicx} % for add picture
%\usepackage{daytime} % for displaying version number and date
\usepackage{datetime} % for current data
\usepackage{indentfirst} % Красная строка
\usepackage[usenames]{color} % for \textcolor
\usepackage{xcolor}
\usepackage{hyperref} % for link
\definecolor{linkcolor}{HTML}{799B03} % цвет ссылок
\definecolor{urlcolor}{HTML}{799B03} % цвет гиперссылок
\hypersetup{colorlinks=true,  linkcolor=linkcolor, urlcolor=urlcolor,}
\usepackage{minted} % for highlight

% seting size page
\voffset=-20mm
\textheight=220mm
\hoffset=-25mm
\textwidth=180mm

\begin{document}

    \begin{center}

    {\Large \bf F.A.Q. для конспектов} \\
    \vspace{0.5em}
    {\large Собрано {\today} в {\currenttime}}

    Эта статья призвана помочь писать конспекты и стандартизировать их написание. Я не буду рассказывать все о LaTeX. Я только опишу инструменты, которыми я советую пользоваться, и которых я надеюсь вам будет достаточно.


    P. S. сложно добиться того, чтобы только из pdf получилась понятная статья, поэтому это статья-пример, которой следует пользоваться так: смотрим код, потом смотрим во что это компилируется.

    \end{center}
\underline{\hbox to 1\textwidth{{ } \hfil{ } \hfil{ } }}

\tableofcontents

\newpage

\section{Установка LaTeX}

\subsection{Linux}

Вам нужно установить:
\begin{enumerate}
    \item texlive-live - сам LaTeX
    \item minted - для вставки исходного кода
    \item pygment - для minted
\end{enumerate}


\textcolor{red}{NB}) Для сборки tex-файла нужно указывать в командной строке флаг \textbf(--shell-escape)


\textcolor{red}{NB}) \href{http://tug.ctan.org/macros/latex/contrib/minted/minted.pdf}{Документация для minted}

\subsection{Win}
    TODO
\section{Работа с Git}
Для того, чтобы работать над конспектом необходимо использовать git. Поэтому я привел несколько ситуации, в которые вы неизбежно попадете.


{\bf Суть:} \url{https://github.com/sorokin/cpp-notes.git} - это основной репрезиторий, на котором лежит релиз конспекта. Чтобы редактировать или создавать статьи, нужно копировать его к себе на git ({\bf forked}), а потом залить в основной репрезиторий ({\bf pull request}).
	\subsection{Начало работы с основным репрезиторем}
	\begin{enumerate}
	\item сделать fork: на \href{https://github.com/sorokin/cpp-notes}{странице} основного репрезитория  вверху справа будет нужная кнопка. (рядом с "Watch"\ )


	Теперь ответвление репрезитория будет у вас на гите. Вы просто работает с ним как со своим.
	\item делаем clone на свой компьютер:
	\begin{minted}{bash}
	git clone git://github.com/my_name/cpp-notes.git
	\end{minted}
	\end{enumerate}
	\subsection{Работа с ответвлением}
	 Тут все просто: меняем, делаем коммит, потом push. Ведь это ваш репрезиторий.


	\textcolor{red}{NB}) Для тех, кому не все просто, отправляю учить \href{https://git-scm.com/book/ru/v1}{основы git}
	\subsection{Залив в основной репрезиторий}
	Сначала нужно залить все изменения на свой git. Потом нужно кнопку \textbf{New pull request} и после этого ваши изменения предложатся создателю репрезетория.

	\subsection{Добавление изменений основного репрезитория в ваш}
	Рассмотрим такую ситуацию: вы работаете над статьей. И в это время, кто-то меняет основной репрезиторий. Как вам получить эти изменения и не потерять то, что вы уже сделали?
	Решение:
	\begin{minted}[breaklines]{bash}
	git remote add sorokin "https://github.com/sorokin/cpp-notes.git" # добавим основной репрезиторий в remote
	git remote -v # убедимся, что он добавился в remote
	git fetch sorokin # сделаем из него ветку
	git branch -v # убедимся, что появилась новая ветка со всем нужными нами изменениями
	git merge sorokin/master #сливаемся с новой веткой
	git push # заливаем все на нас сервер
	\end{minted}


\textcolor{red}{NB}) После того, как мы получили ветку основного репрезитория, мы можем делать с ней все, что хотим. В принципе, можно делать rebase.


\textcolor{red}{NB}) Если кто не силен в ветвлении, смотрите все туже \href{https://git-scm.com/book/ru/v1}{книгу}.
\section{LaTeX}
	TODO

\subsection{Структура}

\begin{enumerate}
    \item Когда вы пишите статью, лучше всего делить ее на секции и подсекции, которые потом будут отражаться в оглавлении. Чем лучше структурирован текст, тем проще ориентироваться в оглавлении.
\end{enumerate}

\subsection{Вставка кода}

Для вставки исходного кода можно использовать разные средства. Я остановился на \textbf{minted}. Мне показалось, что это очень удобная и красивая библиотека. Перейдем к примерам:


Вставим код:
\begin{minted}{c++}

    #include <iostream>

    using namespace std;

    int main() {
        cout << "Hello, world!" << endl;

        return 0;
    }
\end{minted}

Хорошо теперь наведем порядок, написав преамбулу.

\begin{minted}
    [linenos, % нумерация строк
     frame=lines, framesep=2mm, % черта сверху и снизу
     tabsize = 4, % размер табуляции
      breaklines % перенос текста
      ]{c++}

    #include <iostream>

    using namespace std;

    int main() {
        cout << "Hello, world!" << endl; // comment
        return 0;
    }
\end{minted}

Также можно вставлять код прямо в с текст \mintinline{c++}{std::shared_ptr; // smart pointer }. И я рекомендую делать именно так.


\textcolor{red}{NB}) Есть различные стили кода, но стандартная вроде не плоха.

\subsection{Вставка ссылок и гиперссылок}

 \href{http://blog.harrix.org/article/661#h2_2}{Вот статья на эту тему)}


 \url{http://www-sbras.nsc.ru/win/docs/TeX/LaTex2e/hyperref_options.pdf}

 Вот так можно вставлять ссылки.


Также желательно их подсветить в нормальные цвета, что я и сделал в этом документе. Но эти свойства будут прописываться в main.tex, поэтому можно об этом не волноваться.


\section{Советы и пожелания}
 TODO


\end{document}
